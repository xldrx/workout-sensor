When starting a new sport or practicing a new exercise program, it is extremely important to learn basic movements the right way, otherwise a long learning period or serious injuries are not unexpected. This highlights the importance of having a professional personal trainer. The problem is continuous access to such a person is expensive and not always possible.

This problem pushes the market to provide custom-designed sport gadgets. These gadgets are attached to the human body or accessories, then collect various data from built-in sensors. After that companion software analyze this data to evaluate the performance of the user.

Despite the promising outcome, there are some limitations that slow down the vast usage of these devices. First, these gadgets are still not affordable for everyone, and second their application is usually limited to one particular sport or even one particular movement, and can not be extended to others.

One way to overcome these limitations is to use more available devices for the training purposes.

In this work we propose a method to develop a personal trainer using common sensors of smartphone or health tracker. A Hidden Markov Model (HMM) has been trained for each action and classify the input motions by comparing the likelihood of HMMs trained on each action. We would also use a function of likelihood to assess the similarity of an action to the baseline. 