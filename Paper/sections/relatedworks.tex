Classification of human activities is a well-known machine learning problem. Work in \cite{Brand:1997wk} used HMM to classify complex activity. In this work, it is argued that simple HMM has the critical limitation of assuming the system, as a single process with a few states, therefore conventional HMM is not suitable for such detections. Therefore couple HMM is introduced in the work. \cite{Liu:2010cu} used accelerometer for gesture classification. In this method, raw input is quantitated before be used for training an HMM to classify gestures. 
An extension of human activity detection is to detect an activity in a stream. A study of Chambers et. al. in 2004 conducts a set of experiment to explorer and compare performance of different time series representations and distance measures \cite{Chambers:2004cu}. Later in 2008 Ding et. al. tried to segment sport activities using time series analysis. In their work they used HMM on a stream of accelerometer data to detect sport activities \cite{Ding:2008gq}.
