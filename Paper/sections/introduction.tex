- Detect start and end of an activity in a recording
- Find the sport by classification of the recording
- Find the exact movement by classification of the activity
- Score the performance of the user by evaluating the similarity of the recording and the activity baseline

An important part of learning a new sport or exercise is understanding the right technique to correctly perform the physical activities involved. This ensures that people who engage in these activities do not get injured, get the most out of their work, and perform their best. In order to achieve such abilities, one usually hires a professional personal trainer. While this may be the best option, it is not the most affordable one. Many people have the desire and potential to learn a new physical activities but may not have the resources or opportunity to work with a trainer. We hope that developing an application that helps people learn these activities will allow more people to better learn these activities by using a device (such as a smart watch or smart phone) that has an accelerometer. While these devices are not affordable for everyone, they are easier to access than a personal trainer.

	The idea behind the application is that the person will wear the device as they are performing the physical activity. The application will indicate to the user what parts of the movement he or she needs to improve. 	If a user wears these devices while performing an activity, we can track that user’s movement using the accelerometer inside the device. In order to use this data to help the user train for a physical activity, there are a number of obstacles we must overcome. We must determine when the activity in question actually starts and ends during the accelerometer recording so we know what accelerometer features are specific to the actual activity. Once we figure out what features belong to each activity, we can use those features to classify the recording to figure out what activity the user is doing. In order to classify this data we developed a Hidden Markov Model for each activity by obtaining training data for each type of movement. Ideally the training data for these models is from accelerometer recordings of a user that is skilled at the given activity. Once we develop the model we can score the performance of the user when they complete the motion by comparing that motion to the recordings of the skilled user. Dynamic time warping is used to ensure that the the different samples match up.

	
